\subsection{Design Principles}

UPSS is built upon four fundamental design principles that guide all aspects of the framework:

\textbf{Separation of Concerns}: Prompts are treated as configuration data, separate from application logic. This separation enables independent evolution of prompts and code, facilitates security review, and reduces the blast radius of prompt modifications.

\textbf{Defense in Depth}: UPSS implements multiple layered security controls rather than relying on a single protection mechanism. This approach ensures that compromise of one layer does not defeat the entire security posture.

\textbf{Principle of Least Privilege}: Access to prompts and prompt modification capabilities is granted based on legitimate need. Role-based access control ensures that users and systems can only access prompts required for their function.

\textbf{Auditability by Design}: All prompt access, modifications, and usage are logged for security monitoring and compliance. The framework treats audit trails as first-class citizens rather than afterthoughts.

\subsection{Architecture Overview}

The UPSS architecture consists of four primary layers, each providing distinct functionality while maintaining clear interfaces:

\begin{figure}[h]
\centering
\begin{verbatim}
+----------------------------------+
|     Application Layer            |
|  (Business Logic, UI, APIs)      |
+----------------------------------+
           |
           v
+----------------------------------+
|     UPSS Access Layer            |
|  (Loader, Validator, Cache)      |
+----------------------------------+
           |
           v
+----------------------------------+
|     Security Layer               |
|  (Auth, Encryption, Filtering)   |
+----------------------------------+
           |
           v
+----------------------------------+
|     Storage Layer                |
|  (Config Files, Prompts, Audit)  |
+----------------------------------+
\end{verbatim}
\caption{UPSS Architecture Layers}
\label{fig:architecture}
\end{figure}

\textbf{Storage Layer}: Manages persistent storage of configuration files, prompt templates, and audit logs. Supports local filesystem, version control systems, and cloud storage backends.

\textbf{Security Layer}: Implements authentication, authorization, encryption, input validation, and output filtering. This layer enforces security policies before prompts reach the application.

\textbf{Access Layer}: Provides the primary API for loading, validating, and managing prompts. Includes caching, versioning, and metadata management capabilities.

\textbf{Application Layer}: Where business logic resides. Applications interact with UPSS through well-defined APIs without direct access to underlying prompt storage.

\subsection{Core Components}

\textbf{Configuration Loader}: The loader component reads YAML configuration files, parses metadata, and loads prompt content. It validates configuration schema, resolves file paths, and maintains an in-memory representation of the prompt catalog.

\textbf{Prompt Validator}: Validates both configuration structure and prompt content. Checks for required fields, validates data types, ensures file references are valid, and applies custom validation rules.

\textbf{Security Manager}: Enforces access controls, manages encryption keys, implements input sanitization, and filters outputs. The security manager is pluggable, allowing organizations to integrate with existing security infrastructure.

\textbf{Audit Logger}: Records all security-relevant events including prompt access, modifications, validation failures, and security violations. Logs are tamper-evident and support integration with SIEM systems.

\textbf{Cache Manager}: Optimizes performance through intelligent caching of frequently accessed prompts. Implements cache invalidation strategies and supports distributed caching for scaled deployments.

\subsection{Configuration-Based Approach}

UPSS adopts a configuration-first approach with several key advantages:

\textbf{Centralization}: All prompts are cataloged in a single configuration file or directory structure, providing a complete inventory for security review and audit.

\textbf{Versioning}: Configuration files can be version-controlled using standard tools like Git, enabling change tracking, rollback capabilities, and collaborative review processes.

\textbf{Flexibility}: Organizations can define custom metadata fields, validation rules, and security policies tailored to their requirements.

\textbf{Portability}: The YAML-based configuration is human-readable and language-agnostic, supporting implementation across diverse technology stacks.

\subsection{Threat Model}

UPSS addresses threats from three primary adversary classes:

\textbf{External Attackers}: Malicious users attempting to manipulate LLM behavior through prompt injection, extract sensitive information, or bypass access controls.

\textbf{Internal Threats}: Authorized users or compromised accounts attempting to exfiltrate prompts, modify configurations, or abuse system capabilities.

\textbf{Supply Chain Attacks}: Compromised dependencies or infrastructure attempting to inject malicious prompts or tamper with configurations.

\subsection{Assumptions}

UPSS operates under the following security assumptions:

\begin{itemize}
    \item The underlying operating system and filesystem provide basic security guarantees
    \item Cryptographic primitives (encryption, hashing) function as specified
    \item The LLM itself is not directly compromised
    \item Network communications can be secured through TLS
    \item Organizations can establish and enforce access control policies
\end{itemize}
