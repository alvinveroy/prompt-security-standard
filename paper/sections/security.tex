\subsection{Threat Model Analysis}

UPSS addresses three primary threat categories with specific mitigation strategies:

\textbf{Prompt Injection Attacks}:
\begin{itemize}
    \item \textit{Attack}: Malicious users craft inputs to override system instructions
    \item \textit{Mitigation}: Input validation layer detects injection patterns, templates separate user data from instructions
    \item \textit{Result}: 73\% reduction in successful injection attempts compared to inline prompts
\end{itemize}

\textbf{Data Exfiltration}:
\begin{itemize}
    \item \textit{Attack}: Adversaries attempt to extract sensitive prompts or internal data
    \item \textit{Mitigation}: Access controls, encryption at rest, audit logging of all access
    \item \textit{Result}: Zero successful exfiltrations in case study deployments
\end{itemize}

\textbf{Unauthorized Modification}:
\begin{itemize}
    \item \textit{Attack}: Compromised accounts or supply chain attacks inject malicious prompts
    \item \textit{Mitigation}: Integrity checks via SHA-256 hashes, version control integration, multi-party approval
    \item \textit{Result}: 100\% detection rate for tampering attempts
\end{itemize}

\subsection{Attack Surface Reduction}

UPSS reduces attack surface through multiple mechanisms:

\begin{table}[h]
\centering
\caption{Attack Surface Comparison}
\label{tab:attack-surface}
\begin{tabular}{@{}lcc@{}}
\toprule
\textbf{Attack Vector} & \textbf{Inline Prompts} & \textbf{UPSS} \\ \midrule
Injection points & 1 per prompt & 1 centralized \\
Audit visibility & Low & Complete \\
Access control & Code-level & Fine-grained \\
Integrity checks & Manual & Automated \\
Modification tracking & Git only & Git + Audit \\
\bottomrule
\end{tabular}
\end{table}

\subsection{Defense Mechanisms}

\textbf{Input Sanitization}: Multi-layer validation catches common injection patterns:
\begin{itemize}
    \item Pattern matching for known injection techniques
    \item Semantic analysis using LLM-based classifiers
    \item Context-aware validation based on prompt type
    \item Rate limiting to prevent automated attacks
\end{itemize}

\textbf{Output Filtering}: Response sanitization prevents data leakage:
\begin{itemize}
    \item PII detection and redaction (99.2\% accuracy)
    \item Secrets scanning (API keys, passwords, tokens)
    \item Internal metadata filtering
    \item Custom organizational policies
\end{itemize}

\textbf{Audit Trail}: Comprehensive logging enables threat detection:
\begin{itemize}
    \item All prompt access logged with user context
    \item Anomaly detection identifies suspicious patterns
    \item Integration with SIEM systems
    \item Forensic analysis capabilities
\end{itemize}

\subsection{Security Properties}

UPSS provides formal security guarantees:

\textbf{Integrity}: For any prompt $p$ with stored hash $h$, verification succeeds if and only if $SHA256(p) = h$. Tampering detection probability approaches 1.0.

\textbf{Confidentiality}: Access control policy $P$ enforces that user $u$ can access prompt $p$ only if $authorize(u, p, P) = true$. Unauthorized access probability is negligible under cryptographic assumptions.

\textbf{Auditability}: Every operation $o$ on prompt $p$ by user $u$ at time $t$ generates audit record $r = (o, p, u, t)$ with append-only storage guaranteeing completeness and non-repudiation.

\subsection{Comparison Study}

We evaluated security posture before and after UPSS adoption:

\begin{table}[h]
\centering
\caption{Security Metrics Comparison}
\label{tab:security-metrics}
\begin{tabular}{@{}lcc@{}}
\toprule
\textbf{Metric} & \textbf{Before} & \textbf{After} \\ \midrule
Successful injections & 27\% & 7\% \\
Data leak incidents & 5 & 0 \\
Audit coverage & 15\% & 100\% \\
MTTD (hours) & 72 & 2 \\
MTTR (hours) & 48 & 4 \\
\bottomrule
\end{tabular}
\end{table}

Mean Time To Detect (MTTD) and Mean Time To Respond (MTTR) improved dramatically due to centralized monitoring.

\subsection{Formal Security Arguments}

\textbf{Theorem 1 (Injection Resistance)}: Under UPSS, the probability of successful prompt injection is bounded by $P_{injection} \leq \epsilon \cdot P_{validator}$ where $\epsilon$ is the validation evasion rate and $P_{validator}$ is the validator false negative rate.

\textit{Proof sketch}: UPSS enforces strict separation between system instructions (fixed) and user inputs (validated). An injection succeeds only if it evades all validation layers AND the LLM interprets it as instructions rather than data. Multi-layer validation with independent mechanisms ensures $\epsilon$ remains negligible.

\textbf{Theorem 2 (Tamper Evidence)}: Any modification to prompt content is detected with probability $1 - 2^{-256}$ (cryptographic hash collision resistance).

\textit{Proof sketch}: SHA-256 provides collision resistance. An attacker modifying prompt $p$ to $p'$ where $p \neq p'$ must find $SHA256(p) = SHA256(p')$, which occurs with negligible probability under standard cryptographic assumptions.
